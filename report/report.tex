\documentclass[11pt,a4paper]{article}
\usepackage[utf8]{inputenc}
\usepackage{amsmath}
\usepackage{amsfonts}
\usepackage{amssymb}
\author{Kevin Sun - r0653865}
\title{MCS: Rodin project}
\begin{document}
\maketitle
\section{Hours spent}
I spent a total of 11 hours on this project, 5 of which were spent trying to install the needed software on different OS'es.

\section{LTL/CTL statements}
\begin{enumerate}
	\item AG (EF (\{in\_train = TRUE\} \& not [hop\_off] \& not [go\_to\_next]))
	\item G (\{in\_train = TRUE\} $=>$ [go\_to\_next] U [hop\_off])
	\item AF (AG (([hop\_on] or [hop\_off]) \& not [go\_to\_next]))
	\item G (\{current\_station = A\} $=>$ \{current\_station = C\} R \{not (current\_station = F)\})
\end{enumerate}
\subsection{Design decisions}
\begin{enumerate}
	\item "It is always possible ..." means that in every future state, there must be the possibility of getting into a state where the given requirements are fulfilled.
	\item I decided that a ride on a train cannot be infinite, so the use of U forces the passenger to hop off at some point in the future.
	\item "Eventually" is interpreted as: For all paths beginning from the current state, at some point in the path all states of all paths beginning from that point fulfill the given requirement.
	\item /
\end{enumerate}

\end{document}